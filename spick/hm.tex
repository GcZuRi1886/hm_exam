\documentclass[10pt,landscape,a4paper]{article}
\usepackage{multicol}
\usepackage{calc}
\usepackage{ifthen}
\usepackage[landscape]{geometry}
\usepackage{amsmath,amsthm,amsfonts,amssymb}
\usepackage{color,graphicx}
\usepackage{hyperref}

\geometry{top=0.5cm,left=0.5cm,right=0.5cm,bottom=0.5cm}

% Turn off header and footer
\pagestyle{empty}

% Redefine section commands to use less space
\makeatletter
\renewcommand{\section}{\@startsection{section}{1}{0mm}%
                                {-1ex plus -.5ex minus -.2ex}%
                                {0.5ex plus .2ex}%x
                                {\normalfont\large\bfseries}}
\renewcommand{\subsection}{\@startsection{subsection}{2}{0mm}%
                                {-1explus -.5ex minus -.2ex}%
                                {0.5ex plus .2ex}%
                                {\normalfont\normalsize\bfseries}}

\renewcommand{\subsubsection}{\@startsection{subsubsection}{3}{0mm}%
                                {-1ex plus -.5ex minus -.2ex}%
                                {0.5ex plus .2ex}%
                                {\normalfont\small\bfseries}}
\makeatother

\setlength{\parindent}{0pt}
\setlength{\parskip}{0pt plus 0.5ex}

\begin{document}

\begin{center}
	\Large{Höhere Mathematik 1}
\end{center}

\raggedright
\footnotesize
\begin{multicols*}{3}
	\setlength{\premulticols}{1pt}
	\setlength{\postmulticols}{1pt}
	\setlength{\multicolsep}{1pt}
	\setlength{\columnsep}{2pt}

	\section*{Rechnerarithmetik}
	\subsection*{Gleitkommazahlen / Maschinenzahlen}
	\begin{itemize}
		\item Darstellung: \(M = \{x \in \mathbb{R} | x = \pm 0.m_{1}m_{2}m_{3} \ldots m_n \cdot B^{e_{l-1} \ldots e_{1}e_{0} - bias}\} \cup \{0\} \)
		\item Basis \(B\), Mantissenlänge \(n\), Exponentenbereich \([e_{\min}, e_{\max}]\)
		\item Normalisierte Zahl: \(m_1 \neq 0\)
		\item Bias: \(bias = B^{l-1} - 1\) für \(l\)-stellige Exponentendarstellung
	\end{itemize}

	\section*{Fehleranalyse}
	\begin{itemize}
		\item Gerundete Zahl: \(\tilde{x} = rd(x)\)
		\item Absoluter Rundungsfehler: \(|\tilde{x} - x| \leq \frac{1}{2} B^{e - n + 1}\)
		\item Relative Rundungsfehler: \(\left|\frac{\tilde{x} - x}{x}\right| \leq \text{eps}\)
		\item Maschinengenauigkeit: \(\text{eps} = \max_{x \in M} \left|\frac{rd(x) - x}{x}\right| = \frac{1}{2}B^{1-n}\)
	\end{itemize}
	\subsection*{Fehlerfortpflanzung bei Funktionen}
	\begin{itemize}
		\item Absoluter Fehler: \(|f(\tilde{x}) - f(x)| \approx |f'(x)| \cdot |\tilde{x} - x|\)
		\item Relativer Fehler: \(\left|\frac{f(\tilde{x}) - f(x)}{f(x)}\right| \approx \left|\frac{x \cdot f'(x)}{f(x)}\right| \cdot \left|\frac{\tilde{x} - x}{x}\right|\)
		\item Kondition (Je kleiner desto besser): \(K = \left|\frac{x \cdot f'(x)}{f(x)}\right|\)
	\end{itemize}

	\section*{Nullstellenprobleme}
  
  \subsection*{Fixpunktiteration}
  \begin{itemize}
    \item Fixpunktgleichung: \(F(x) = x\)
    \item Fixpunkte $(\overline{x})$ sind Nullstellen von \(f(x) = F(x) - x\)
    \item Iterationsvorschrift: \(x_{n+1} = F(x_n)\)
    \item Konvergenz: \(|F'(\overline{x})| < 1\) $\rightarrow$ anziehender Fixpunkt
    \item Divergenz: \(|F'(\overline{x})| > 1\) $\rightarrow$ abstoßender Fixpunkt
  \end{itemize}
  \subsubsection*{Banach'scher Fixpunktsatz}
  \begin{itemize}
    \item Sei \(F: [a,b] \rightarrow [a,b]\) und es existiert eine Konstante \(0 < \alpha < 1\) mit 
      \begin{equation*}
        |F(x) - F(y)| \leq \alpha | x - y |  \quad \forall x,y \in [a,b]
      \end{equation*}
    \item Dann besitzt \(F\) genau einen Fixpunkt \(\overline{x} \in [a,b]\) und die Iteration \(x_{n+1} = F(x_n)\) konvergiert für jeden Startwert \(x_0 \in [a,b]\) gegen \(\overline{x}\). (Lipschitz-Stetigkeit mit Lipschitz-Konstante \(\alpha\))
    \item a-priori-Fehlerabschätzung:
      \begin{equation*}
        |x_n - \overline{x}| \leq \frac{\alpha^n}{1 - \alpha} |x_1 - x_0|
      \end{equation*}
    \item a-posteriori-Fehlerabschätzung:
      \begin{equation*}
        |x_n - \overline{x}| \leq \frac{\alpha}{1 - \alpha} |x_n - x_{n-1}|
      \end{equation*}
    \item Umformung: $\left | \frac{F(x) - F(y)}{x - y}{x - y} \right | \leq \alpha$ 
  \end{itemize}
  
  \columnbreak

  \subsection*{Newton-Verfahren}
  \begin{equation*}
    x_{n+1} = x_n - \frac{f(x_n)}{f'(x_n)}
  \end{equation*}
  Für den Startwert $x_0$ muss gelten, dass $f(x_0) \ne 0$ und $f'(x_0) \ne 0$.
  Ein geeigneter Startwert kann durch die Nullstellenbestimmung der Ableitungsfunktion gefunden werden.
  \subsubsection*{Vereinfachte Newton-Verfahren}
  \begin{itemize}
    \item Fixierter Ableitungswert: \(x_{n+1} = x_n - \frac{f(x_n)}{f'(x_0)}\)
    \item Sekantenverfahren: \(x_{n+1} = x_n - f(x_n) \cdot \frac{x_n - x_{n-1}}{f(x_n) - f(x_{n-1})}\)
  \end{itemize}

  \subsection*{Fehlerabschätzung}
  \subsubsection*{Konvergenzordnung}
  \begin{itemize}
    \item Sei \((x_n)\) eine konvergierende Folge gegen \(\overline{x}\)
    \item Die Folge hat die Konvergenzordnung \(q\), wenn eine Konstante \(c > 0\) existiert, so dass
      \begin{equation*}
        | x_{n+1} - \overline{x} | \leq c \cdot | x_n - \overline{x} |^q
      \end{equation*}
    \item \(q = 1\) lineare Konvergenz, \(q = 2\) quadratische Konvergenz
    
  \end{itemize}
  \subsubsection*{Nullstellensatz von Bolzano}
  \begin{itemize}
    \item Sei \(f: [a,b] \rightarrow \mathbb{R}\) stetig mit \(f(a) \cdot f(b) < 0\) oder \(f(a) \geq 0 \geq f(b)\) 
    \item Dann existiert mindestens ein \(\overline{x} \in (a,b)\) mit \(f(\overline{x}) = 0\)
  \end{itemize}
  
  \section*{Lineare Gleichungssysteme}
  \subsection*{LR-Zerlegung}
  \begin{itemize}
    \item Zerlegung einer Matrix \(A\) in das Produkt \(A = L \cdot R\)
    \item \(L\): untere Dreiecksmatrix mit Einsen auf der Diagonale
    \item \(R\): obere Dreiecksmatrix
    \item Lösen von \(A \cdot x = b\) durch Lösen von \(L \cdot y = b\) und \(R \cdot x = y\)
    \item Voraussetzung: Keine Null-Pivotelemente während der Zerlegung / kein Zeilentausch notwendig (sonst \(P \cdot A = L \cdot R\))
  \end{itemize}
  \subsubsection*{Beispiel}
  \begin{equation*}
    A = \begin{pmatrix}
      -1 & 1 & 1 \\
      1 & -3 & -2 \\
      5 & 1 & 4
    \end{pmatrix} = \underbrace{\begin{pmatrix}
      1 & 0 & 0 \\
      -1 & 1 & 0 \\
      -5 & -3 & 1
    \end{pmatrix}}_{L} \cdot \underbrace{\begin{pmatrix}
      -1 & 1 & 1 \\
      0 & -2 & -1 \\
      0 & 0 & 6
    \end{pmatrix}}_{R}
  \end{equation*}
  \subsubsection*{Permutationsmatrix}
  \begin{itemize}
    \item Vertauscht Zeilen oder Spalten einer Matrix
    \item \(P \cdot A\): Vertauschen der Zeilen von \(A\)
    \item \(A \cdot P\): Vertauschen der Spalten von \(A\)
    \item \( P_1, P_2\) sind Permutationsmatrizen $\Rightarrow$ \(P = P_1 \cdot P_2\) ist auch eine Permutationsmatrix
    \item \(P^{-1} = P^T\)
  \end{itemize}

  \columnbreak

  \subsection*{QR-Zerlegung}
  \begin{itemize}
    \item Zerlegung einer Matrix \(A\) in das Produkt \(A = Q \cdot R\)
    \item \(Q\): orthogonale Matrix (\(Q^T \cdot Q = I\))
    \item \(R\): obere Dreiecksmatrix
  \end{itemize}
  \subsubsection*{Householder-Matrizen}
  \begin{itemize}
    \item \(H\) ist orthogonal und symmetrisch (\(H^T = H\))
    \item \(H\) spiegelt an der Hyperebene orthogonal zu \(u\)
    \item \( H = H^T = H^{-1} \Rightarrow H \cdot H = I_n \)
    \item Wenn \(u\) ein normierter Vektor ist, dann gilt:
      \begin{equation*}
        H = I_n - 2 u u^T
      \end{equation*}
  \end{itemize}
  \subsubsection*{QR-Zerlegung mit Householder-Matrizen}
  \begin{equation*}
    A = \begin{pmatrix}
      1 & 2 & -1 \\
      4 & -2 & 6 \\
      3 & 1 & 0
    \end{pmatrix}, b = \begin{pmatrix}
      9 \\
      -4 \\
      9
    \end{pmatrix}
  \end{equation*}
  \begin{enumerate}
    \item \(v_1 = a_1 + sign(a_{11}) \cdot |a_1| \cdot e_1\)
    \item \(u_1 = \frac{v_1}{||v_1||_2}\)
    \item \(H_1 = I - 2 u_1 u_1^T\)
    \item \(A^{(1)} = H_1 \cdot A\), \(b^{(1)} = H_1 \cdot b\)
  \end{enumerate}
  \begin{equation*}
    A^{(1)} = \begin{pmatrix}
      -5.0990 & 0.5883 & -4.5107 \\
      0 & -2.9258 & 3.6976 \\
      0 & 0.3056 & -1.7268
    \end{pmatrix}
  \end{equation*}
  Rekursiv mit \(A^{(1)}\) (2$\times$2-Matrix) durchführen bis \(A^{(n-1)}\) obere Dreiecksmatrix ist.

  \subsection*{Vektor- und Matrixnormen}
  \begin{itemize}
    \item Vektornormen:
      \begin{equation*}
        \text{1-Norm, Summennorm: } \|x\|_1 = \sum_{i=1}^{n} |x_i|
      \end{equation*}
      \begin{equation*}
        \text{2-Norm, euklidische Norm: } \|x\|_2 = \sqrt{\sum_{i=1}^{n} |x_i|^2}
      \end{equation*}
      \begin{equation*}
        \text{Unendlich-Norm, Maximumnorm: } \|x\|_{\infty} = \max_{1 \leq i \leq n} |x_i|
      \end{equation*}
   \item Matrixnormen (n$\times$m-Matrizen):
      \begin{equation*}
        \text{1-Norm, Spaltensummennorm: } \|A\|_1 = \max_{1 \leq j \leq m} \sum_{i=1}^{n} |a_{ij}|
      \end{equation*}
      \begin{equation*}
        \text{2-Norm, Spektralnorm: } \|A\|_2 = \sqrt{\rho(A^T A)}
      \end{equation*}
      \begin{equation*}
        \text{Unendlich-Norm, Zeilensummennorm: } \|A\|_{\infty} = \max_{1 \leq i \leq n} \sum_{j=1}^{m} |a_{ij}|
      \end{equation*}
  \end{itemize}
  \subsubsection*{Spektralradius}
  \begin{itemize}
    \item \(\rho(A) = \max \{ |\lambda| | \lambda \text{ ist Eigenwert von } A \}\)
    \item Für jede Matrixnorm gilt: \(\rho(A) \leq \|A\|\)
  \end{itemize}
  
  \pagebreak

  \subsection*{Fehlerrechnung Vektoren}
  \begin{itemize}
    \item $|| . ||$: gewählte Matrixnorm
    \item Konditionszahl: \(cond(A) = \|A\| \cdot \|A^{-1}\|\)
    \item Abschätzung des relativen Lösungsfehlers:
      \begin{equation*}
        \frac{\| \tilde{x} - x \|}{\| x \|} \leq cond(A) \cdot \frac{\| \tilde{b} - b \|}{\| b \|} \text{, falls } || b || \neq 0
      \end{equation*}
    \item Je größer \(cond(A)\), desto schlechter ist das LGS konditioniert
  \end{itemize}
  \subsection*{Fehlerrechnung Matrizen}
  \begin{itemize}
    \item Gegeben: \( \tilde{A} = A + E \) mit \(E\) als Störmatrix ($E = A - \tilde{A}$)
    \item Abschätzung des relativen Lösungsfehlers:
      \begin{equation*}
        \frac{\| \tilde{x} - x \|}{\| x \|} \leq \frac{cond(A)}{1 - cond(A) \cdot \frac{\| E \|}{\| A \|}} \cdot \left ( \frac{\| E \|}{\| A \|} + \frac{\|b - \tilde{b}\|}{\|b\|} \right )
      \end{equation*}
    \item Gilt nur, wenn \(cond(A) \cdot \frac{\| E \|}{\| A \|} < 1\)
  \end{itemize}
  \subsection*{Aufwandsabschätzung}
  \begin{itemize}
    \item LR-Zerlegung: \(\frac{2}{3} n^3\) Operationen
    \item Vorwärts- und Rückwärtseinsetzen: \(2 n^2\) Operationen
    \item QR-Zerlegung mit Householder: \(2 n^3\) Operationen
    \item Vorwärts- und Rückwärtseinsetzen: \(2 n^2\) Operationen
    \item Allgemein:
      \begin{equation*}
        \sum_{i=1}^{n}i = \frac{n(n+1)}{2} \approx \frac{n^2}{2}
      \end{equation*}
      und
      \begin{equation*}
        \sum_{i=1}^{n}i^2 = \frac{1}{6} n + \frac{1}{2} n^2 + \frac{1}{3} n^3 \approx \frac{1}{3} n^3
      \end{equation*}
  \end{itemize}
  \section*{Iterative Verfahren für LGS}
  \subsection*{Jacobi-Verfahren}
  \begin{itemize}
    \item Zerlegung: 
      D ist die Diagonalmatrix von A, L die untere Dreiecksmatrix und R die obere Dreiecksmatrix
      \begin{align*}
        Ax &= b \\
        (L + D + R)x &= b \\
        Dx &= - (L + R)x + b \\
        x &= D^{-1} (L + R)x + D^{-1} b
      \end{align*}
    \item Iterationsvorschrift: \(x^{(k+1)} = D^{-1} (L + R) x^{(k)} + D^{-1} b\)
    \item Iterationsmatrix: \(B_J = D^{-1} (L + R)\)
  \end{itemize}
  \subsection*{Gauß-Seidel-Verfahren}
  \begin{itemize}
    \item Iterationsvorschrift: \(x^{(k+1)} = (D - L)^{-1} R x^{(k)} + (D - L)^{-1} b\)
    \item Iterationsmatrix: \(B_{GS} = (D - L)^{-1} R\)
  \end{itemize}
  \subsection*{Konvergenz und Abschätzung}
  \begin{itemize}
    \item Beide Verfahren konvergieren, wenn \(\|B\| < 1\)
    \item Symmetrisch positiv definite Matrizen (nur Gauß-Seidel)
    \item Siehe Fixpunktiteration (\(\alpha = \|B\|\))
  \end{itemize}

  \columnbreak
  
  \subsubsection*{Diagonaldominanz}
  \begin{itemize}
    \item Eine Matrix ist diagonaldominant wenn eines der folgenden Kriterien erfüllt ist:
      \begin{itemize}
        \item \(\forall i = 1, \ldots,n: |a_{ii}| > \sum_{j = 1, j \neq i} |a_{i,j}|\) (Zeilensummenkriterium)
        \item \(\forall j = 1, \ldots,n: |a_{jj}| > \sum_{i = 1, i \neq j} |a_{i,j}|\) (Spaltensummenkriterium)
      \end{itemize}
    \item Beide Verfahren konvergieren für diagonaldominante Matrizen
  \end{itemize}

  \section*{Komplexe Zahlen}
  \subsection*{Darstellung}
  \begin{itemize}
    \item Normalform / kartesische Form: \(z = x + iy\), \(x,y \in \mathbb{R}\)
    \item Polardarstellung / Trigonometrische Form: \(z = r (\cos \varphi + i \sin \varphi)\)
    \item Eulersche Darstellung: \(z = r e^{i \varphi}\)
    \item Betrag: \(|z| = r = \sqrt{x^2 + y^2}\)
    \item Argument: \(\varphi = \arg(z) = \tan^{-1}(\frac{y}{x}) = \cos^{-1}(\frac{x}{r}) = \sin^{-1}(\frac{y}{r})\)
  \end{itemize}

  \subsection*{Rechenregeln}
  \begin{itemize}
    \item Addition: \(z_1 + z_2 = (x_1 + x_2) + i(y_1 + y_2)\)
    \item Multiplikation: \(z_1 \cdot z_2 = (x_1 + i y_1)(x_2 + i y_2)\)
    \item Multiplikation in Polardarstellung:
      \begin{equation*}
        z_1 \cdot z_2 = r_1 r_2 [\cos(\varphi_1 + \varphi_2) + i \sin(\varphi_1 + \varphi_2)]
      \end{equation*}
    \item Division: 
      \begin{equation*}
        \frac{z_1}{z_2} = \frac{z_1 \cdot z_2^*}{z_2 \cdot z_2^*}, \quad z_2 \neq 0
      \end{equation*}
    \item Division in Polardarstellung:
      \begin{equation*}
        \frac{z_1}{z_2} = \frac{r_1}{r_2} [\cos(\varphi_1 - \varphi_2) + i \sin(\varphi_1 - \varphi_2)], \quad z_2 \neq 0
      \end{equation*}
    \item Potenzieren in der Polardarstellung:
      \begin{equation*}
        z^n = r^n [\cos(n \varphi) + i \sin(n \varphi)] = r^n e^{i n \varphi}
      \end{equation*}
  \end{itemize}
  \subsection*{Wurzeln}
  \begin{itemize}
    \item \(z = r (\cos \varphi + i \sin \varphi)\) = \(r e^{i \varphi}\)
    \item \(n\)-te Wurzel:
      \begin{equation*}
        z_k = r^{1/n} \left[ \cos\left( \frac{\varphi + 2k\pi}{n} \right) + i \sin\left( \frac{\varphi + 2k\pi}{n} \right) \right] = r e^{i \frac{\varphi + 2k\pi}{n}} 
      \end{equation*}
      \begin{equation*}
        k = 0, 1, \ldots, n-1
      \end{equation*}
  \end{itemize}
\end{multicols*}
\end{document}
