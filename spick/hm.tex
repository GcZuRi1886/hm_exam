\documentclass[10pt,landscape,a4paper]{article}
\usepackage{multicol}
\usepackage{calc}
\usepackage{ifthen}
\usepackage[landscape]{geometry}
\usepackage{amsmath,amsthm,amsfonts,amssymb}
\usepackage{color,graphicx}
\usepackage{hyperref}

\geometry{top=0.5cm,left=0.5cm,right=0.5cm,bottom=0.5cm}

% Turn off header and footer
\pagestyle{empty}

% Redefine section commands to use less space
\makeatletter
\renewcommand{\section}{\@startsection{section}{1}{0mm}%
                                {-1ex plus -.5ex minus -.2ex}%
                                {0.5ex plus .2ex}%x
                                {\normalfont\large\bfseries}}
\renewcommand{\subsection}{\@startsection{subsection}{2}{0mm}%
                                {-1explus -.5ex minus -.2ex}%
                                {0.5ex plus .2ex}%
                                {\normalfont\normalsize\bfseries}}

\renewcommand{\subsubsection}{\@startsection{subsubsection}{3}{0mm}%
                                {-1ex plus -.5ex minus -.2ex}%
                                {0.5ex plus .2ex}%
                                {\normalfont\small\bfseries}}
\makeatother

\setlength{\parindent}{0pt}
\setlength{\parskip}{0pt plus 0.5ex}

\begin{document}

\begin{center}
	\Large{Höhere Mathematik 1}
\end{center}

\raggedright
\footnotesize
\begin{multicols*}{3}
	\setlength{\premulticols}{1pt}
	\setlength{\postmulticols}{1pt}
	\setlength{\multicolsep}{1pt}
	\setlength{\columnsep}{2pt}

	\section*{Rechnerarithmetik}
	\subsection*{Gleitkommazahlen / Maschinenzahlen}
	\begin{itemize}
		\item Darstellung: \(M = \{x \in \mathbb{R} | x = \pm 0.m_{1}m_{2}m_{3} \ldots m_n \cdot B^{e_{l-1} \ldots e_{1}e_{0} - bias}\} \cup \{0\} \)
		\item Basis \(B\), Mantissenlänge \(n\), Exponentenbereich \([e_{\min}, e_{\max}]\)
		\item Normalisierte Zahl: \(m_1 \neq 0\)
		\item Bias: \(bias = B^{l-1} - 1\) für \(l\)-stellige Exponentendarstellung
	\end{itemize}

	\section*{Fehleranalyse}
	\begin{itemize}
		\item Gerundete Zahl: \(\tilde{x} = rd(x)\)
		\item Absoluter Rundungsfehler: \(|\tilde{x} - x| \leq \frac{1}{2} B^{e - n + 1}\)
		\item Relative Rundungsfehler: \(\left|\frac{\tilde{x} - x}{x}\right| \leq \text{eps}\)
		\item Maschinengenauigkeit: \(\text{eps} = \max_{x \in M} \left|\frac{rd(x) - x}{x}\right| = \frac{1}{2}B^{1-n}\)
	\end{itemize}
	\subsection*{Fehlerfortpflanzung bei Funktionen}
	\begin{itemize}
		\item Absoluter Fehler: \(|f(\tilde{x}) - f(x)| \approx |f'(x)| \cdot |\tilde{x} - x|\)
		\item Relativer Fehler: \(\left|\frac{f(\tilde{x}) - f(x)}{f(x)}\right| \approx \left|\frac{x \cdot f'(x)}{f(x)}\right| \cdot \left|\frac{\tilde{x} - x}{x}\right|\)
		\item Kondition (Je kleiner desto besser): \(K = \left|\frac{x \cdot f'(x)}{f(x)}\right|\)
	\end{itemize}

	\section*{Nullstellenprobleme}
  
  \subsection*{Fixpunktiteration}
  \begin{itemize}
    \item Fixpunktgleichung: \(F(x) = x\)
    \item Fixpunkte $(\overline{x})$ sind Nullstellen von \(f(x) = F(x) - x\)
    \item Iterationsvorschrift: \(x_{n+1} = F(x_n)\)
    \item Konvergenz: \(|F'(\overline{x})| < 1\) $\rightarrow$ anziehender Fixpunkt
    \item Divergenz: \(|F'(\overline{x})| > 1\) $\rightarrow$ abstoßender Fixpunkt
  \end{itemize}
  \subsubsection*{Banach'scher Fixpunktsatz}
  \begin{itemize}
    \item Sei \(F: [a,b] \rightarrow [a,b]\) und es existiert eine Konstante \(0 < \alpha < 1\) mit 
      \begin{equation*}
        |F(x) - F(y)| \leq \alpha | x - y |  \quad \forall x,y \in [a,b]
      \end{equation*}
    \item Dann besitzt \(F\) genau einen Fixpunkt \(\overline{x} \in [a,b]\) und die Iteration \(x_{n+1} = F(x_n)\) konvergiert für jeden Startwert \(x_0 \in [a,b]\) gegen \(\overline{x}\). (Lipschitz-Stetigkeit mit Lipschitz-Konstante \(\alpha\))
    \item a-priori-Fehlerabschätzung:
      \begin{equation*}
        |x_n - \overline{x}| \leq \frac{\alpha^n}{1 - \alpha} |x_1 - x_0|
      \end{equation*}
    \item a-posteriori-Fehlerabschätzung:
      \begin{equation*}
        |x_n - \overline{x}| \leq \frac{\alpha}{1 - \alpha} |x_n - x_{n-1}|
      \end{equation*}
    \item Umformung: $\left | \frac{F(x) - F(y)}{x - y}{x - y} \right | \leq \alpha$ 
  \end{itemize}
  
  \columnbreak

  \subsection*{Newton-Verfahren}
  \begin{equation*}
    x_{n+1} = x_n - \frac{f(x_n)}{f'(x_n)}
  \end{equation*}
  Für den Startwert $x_0$ muss gelten, dass $f(x_0) \ne 0$ und $f'(x_0) \ne 0$.
  Ein geeigneter Startwert kann durch die Nullstellenbestimmung der Ableitungsfunktion gefunden werden.
  \subsubsection*{Vereinfachte Newton-Verfahren}
  \begin{itemize}
    \item Fixierter Ableitungswert: \(x_{n+1} = x_n - \frac{f(x_n)}{f'(x_0)}\)
    \item Sekantenverfahren: \(x_{n+1} = x_n - f(x_n) \cdot \frac{x_n - x_{n-1}}{f(x_n) - f(x_{n-1})}\)
  \end{itemize}

  \subsection*{Fehlerabschätzung}
  \subsubsection*{Konvergenzordnung}
  \begin{itemize}
    \item Sei \((x_n)\) eine konvergierende Folge gegen \(\overline{x}\)
    \item Die Folge hat die Konvergenzordnung \(q\), wenn eine Konstante \(c > 0\) existiert, so dass
      \begin{equation*}
        | x_{n+1} - \overline{x} | \leq c \cdot | x_n - \overline{x} |^q
      \end{equation*}
    \item \(q = 1\) lineare Konvergenz, \(q = 2\) quadratische Konvergenz
    
  \end{itemize}
  \subsubsection*{Nullstellensatz von Bolzano}
  \begin{itemize}
    \item Sei \(f: [a,b] \rightarrow \mathbb{R}\) stetig mit \(f(a) \cdot f(b) < 0\) oder \(f(a) \geq 0 \geq f(b)\) 
    \item Dann existiert mindestens ein \(\overline{x} \in (a,b)\) mit \(f(\overline{x}) = 0\)
  \end{itemize}
\end{multicols*}
\end{document}
